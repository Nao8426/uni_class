\section{感想}
\begin{itemize}
    \setlength{\itemsep}{5mm}
    \item 濱崎 直紀(馬場口研究室)\\
    担当:プログラミング(サンプルプログラム),レポート作成\par
    \vspace{2mm}
    \quad
    研究の一環などで画像から特徴などを抽出することはあったが,基本的にOpenCVなどの画像処理系ライブラリを用いていたため,実際にその内部で行われていることをしっかり理解していなかった.
    しかし,今回の課題があったことで,数多くある手法の一つではあるが,FASTの内部で行われていることの理解を深めることができた.
    ライブラリの存在により実際に行われていることが曖昧になりがちだったが,そのアルゴリズムを知ることの重要さを感じた.\par\quad
    また,FASTのプログラムを実際に書くことに関してはほぼ貢献できなかったという反省点があるので,次回の課題の際にはグループの一員としての役割をしっかりと果たせるように努めたい.
    \item 永井 智之(井上研究室)\\
    担当:プログラミング(ガウスノイズの実装)\par
    \vspace{2mm}
    \quad
    ガウスノイズの実装を行なったが,FASTの方での実装は行なっておらず,何か結果を示すことは出来なかった.
    問題点として,目標をしっかりと定めずに課題を行なっていたことが挙げられる.\par\quad
    また,FASTの方は中尾くんに任せっきりだったので,次回の課題では班に貢献できるようにしたい.
    \item 中尾 文亮(八木研究室)\\
    担当:プログラミング(FASTの実装)\par
    \vspace{2mm}
    \quad
    今回FASTプログラムを実装したが,高速化のプログラムを省略してしまったり,周囲のピクセルとの輝度比較のプログラムで冗長な書き方をしてしまうなど,主に「処理速度」の点で妥協ののこる実装になってしまった.
    次回の課題では,ただ動けばいいということではなく,速さも求めたプログラムを書きたい.
\end{itemize}
