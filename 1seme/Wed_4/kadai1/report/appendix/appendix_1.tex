\section{プログラム1}
\begin{lstlisting}[caption=FAST]
    import cv2
    import numpy as np
    
    class FAST:
    
      def __init__(self, t, n):
        self.t = t
        self.n = n 
    
      # カラー画像を引数として、FASTで抽出した特徴点座標をタプルの配列で返す
      def get_img_feature(self, img):
    
        img_feature= []
    
        img = cv2.cvtColor(img, cv2.COLOR_BGR2GRAY)
        height, width = img.shape
    
        for x in range(width):
          for y in range(height):
    
            #上下左右15ピクセルは処理しない(接合時のことを考慮)
            if (
                 x in range(0,15) 
              or x in range(width-15,width) 
              or y in range(0,15) 
              or y in range(height-15,height)
              ):
                continue
    
            standard_brightness = img[y][x] #基準となる明るさ
    
            surrounding_pixels = np.array(
              [
                int(img[y-3][x]),
                int(img[y-3][x+1]),
                int(img[y-2][x+2]),
                int(img[y-1][x+3]),
                int(img[y][x+3]),
                int(img[y+1][x+3]),
                int(img[y+2][x+2]),
                int(img[y+3][x+1]),
                int(img[y+3][x]),
                int(img[y+3][x-1]),
                int(img[y+2][x-2]),
                int(img[y+1][x-3]),
                int(img[y][x-3]),
                int(img[y-1][x-3]),
                int(img[y-2][x-2]),
                int(img[y-3][x-1])
              ]
            )
    
            # #現状これがないほうが早く動くんだよなぁ..
            # #高速化プログラム(本来n>12の時適用)#######################################################
            # brightness_status_array = [0,0]
    
            # for i in [0,4,8,12]:
            #   brightness_status = self.__get_brightness_status(brightness_around_array[i], brightness, self.t)
            #   if brightness_status == "bright":
            #     brightness_status_array[0] += 1 
            #   elif brightness_status == "dark":
            #     brightness_status_array[1] += 1
    
            #   #明るい点が2点の場合break
            #   if not (brightness_status_array[0]>2 or brightness_status_array[1]>2):
            #     continue
            # ######################################################################################
    
            consecutive_num_array = [] #明るさ判断が連続して一致した数の格納用
            consecutive_num_el = 0 #明るさ判断が連続して一致した数の一時保存用
            brightness_status = 0
            brightness_status_before = 0
    
            #0部分のbrightness statusをここで定義
            brightness_status_before = self.__get_brightness_status(surrounding_pixels[0], standard_brightness)
    
            for i in range(16):
    
              #0部分のbrightness statusをここで定義
              brightness_status = self.__get_brightness_status(surrounding_pixels[i], standard_brightness)
    
              #前の要素の明るさが中間値でなく、かつ前の要素と一致する場合
              if brightness_status != 0 and brightness_status_before == brightness_status:
    
                consecutive_num_el += 1
    
              else: #前の要素と明るさに関する要件が一致しない
    
                consecutive_num_array.append(consecutive_num_el)
                consecutive_num_el = 1
              
              #brightness_statusの更新
              brightness_status_before = brightness_status
            
            #最終結果および、最初と最後の足し合わせを末尾に追加
            consecutive_num_array.append(consecutive_num_el)
            consecutive_num_array.append( consecutive_num_array[0] + consecutive_num_array[-1] )
    
            if np.amax(consecutive_num_array) > self.n:
              img_feature.append((x,y))
    
        print("feature detect process finished")
        return img_feature
    
    
      #2画像とその特徴点からもっともマッチした組み合わせを返す
      #(この関数は回転対応のためにちょっと書き換える必要あり)
      def get_best_match_feature(self, img1, img1_features, img2, img2_features):
    
        img1_best_feature = (0,0)
        img2_best_feature = (0,0)
        min_diff = 1000000000
    
        for feature1 in img1_features:
    
          x1,y1 = feature1
    
          for feature2 in img2_features:
    
            x2,y2 = feature2
    
            img1_fraction = img1[y1-10:y1+10, x1-10:x1+10].astype(np.int8).flatten()
            img2_fraction = img2[y2-10:y2+10, x2-10:x2+10].astype(np.int8).flatten()
    
            diff = np.sum(np.abs(img1_fraction-img2_fraction))
    
            if min_diff > diff:
                min_diff = diff
                img1_best_feature = (x1,y1)
                img2_best_feature = (x2,y2)
    
        return (img1_best_feature, img2_best_feature)
    
      #brightness_status取得部分
      def __get_brightness_status (self, target, brightness):
        if target > brightness+self.t:
          return 1
        elif target < brightness-self.t:
          return -1
        else:
          return 0
          
\end{lstlisting}
