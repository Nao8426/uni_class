\section*{課題2}
課題1で作成したプログラムを応用し, 数字0~9の学習用画像を用いて,手書き文字(モノクロ画像)を認識する識別器を作成し, 手書き文字を複数含む画像から各数字を検出するプログラムを作成せよ.

\subsection*{SVMにおける多クラス分類}
SVMは基本的に2クラス分類の手法であるため,多クラス分類に用いるには工夫が必要となる.
工夫の方法は幾つかあるが,その中でもよく用いられる"one-versus-the-rest"と"one-versus-one"について説明を行う.\par
\begin{itemize}
    \setlength{\itemsep}{3mm}
    \item one-versus-the-rest\par
    \vspace{1mm}
    \quad
    ある特定のクラスに入るか,他の$k-1$個のクラスのどれかに入るかの2クラス識別器を作成して識別を行う.
    最終的な識別はそれらの識別結果による多数決で行う.
    識別器は$k$個必要.
    \item one-versus-one\par
    \vspace{1mm}
    \quad
    ある特定のクラスに入るか,別の特定のクラスに入るかの2クラス識別器を作成して識別を行う.
    最終的な識別はそれらの識別結果による多数決で行う.
    識別器は$\frac{k(k−1)}{2}$個必要.
\end{itemize}
本実験ではone-versus-oneによる識別を行った.

\subsection*{特徴量}
特徴量として次元数削減のため,画像を縮小することを考えた.
画像縮小を行わなかった場合と行った場合におけるMNISTのいづれか2つの数字の画像をSVMで識別した結果は以下の通りである.
\begin{table}[H]
    \begin{center}
        \begin{tabular}{|c|c||c|c||c|c||c|c||c|c|}
            \hline
            Target & Accuracy & Target & Accuracy & Target & Accuracy & Target & Accuracy & Target & Accuracy\\
            \hline \hline
            0-1 & 99.83\% & 1-2 & 98.88\% & 2-4 & 97.93\% & 3-7 & 97.87\% & 5-7 & 99.09\%\\
            0-2 & 98.70\% & 1-3 & 98.97\% & 2-5 & 96.89\% & 3-8 & 94.42\% & 5-8 & 94.37\%\\
            0-3 & 99.23\% & 1-4 & 99.40\% & 2-6 & 97.65\% & 3-9 & 97.64\% & 5-9 & 97.89\%\\
            0-4 & 99.55\% & 1-5 & 99.26\% & 2-7 & 97.87\% & 4-5 & 98.53\% & 6-7 & 99.86\%\\
            0-5 & 97.99\% & 1-6 & 99.82\% & 2-8 & 94.37\% & 4-6 & 98.68\% & 6-8 & 98.12\%\\
            0-6 & 98.40\% & 1-7 & 99.18\% & 2-9 & 98.31\% & 4-7 & 98.26\% & 6-9 & 99.73\%\\
            0-7 & 99.56\% & 1-8 & 97.54\% & 3-4 & 99.08\% & 4-8 & 98.92\% & 7-8 & 98.85\%\\
            0-8 & 98.61\% & 1-9 & 99.45\% & 3-5 & 93.30\% & 4-9 & 94.19\% & 7-9 & 92.50\%\\
            0-9 & 99.32\% & 2-3 & 95.65\% & 3-6 & 99.37\% & 5-6 & 96.96\% & 8-9 & 97.29\%\\
            \hline
        \end{tabular}
    \end{center}
    \begin{center}
        \begin{tabular}{|c|c||c|c||c|c||c|c||c|c|}
            \hline
            Target & Accuracy & Target & Accuracy & Target & Accuracy & Target & Accuracy & Target & Accuracy\\
            \hline \hline
            0-1 & 99.79\% & 1-2 & 98.17\% & 2-4 & 96.98\% & 3-7 & 97.61\% & 5-7 & 99.04\%\\
            0-2 & 98.12\% & 1-3 & 98.68\% & 2-5 & 96.69\% & 3-8 & \textcolor{red}{52.75\%} & 5-8 & \textcolor{red}{53.09\%}\\
            0-3 & 98.83\% & 1-4 & 99.32\% & 2-6 & 96.61\% & 3-9 & 97.27\% & 5-9 & 96.89\%\\
            0-4 & 99.41\% & 1-5 & 99.40\% & 2-7 & 97.61\% & 4-5 & 98.07\% & 6-7 & 99.73\%\\
            0-5 & 97.40\% & 1-6 & 99.64\% & 2-8 & 95.27\% & 4-6 & 98.84\% & 6-8 & 98.52\%\\
            0-6 & 98.45\% & 1-7 & 99.21\% & 2-9 & 97.85\% & 4-7 & 97.65\% & 6-9 & 99.54\%\\
            0-7 & 99.42\% & 1-8 & 96.23\% & 3-4 & 99.30\% & 4-8 & 98.48\% & 7-8 & 98.98\%\\
            0-8 & 97.95\% & 1-9 & 99.35\% & 3-5 & \textcolor{red}{64.79\%} & 4-9 & \textcolor{red}{56.80\%} & 7-9 & \textcolor{red}{51.31\%}\\
            0-9 & 99.19\% & 2-3 & \textcolor{red}{55.05\%} & 3-6 & 99.48\% & 5-6 & 97.22\% & 8-9 & 97.10\%\\
            \hline
        \end{tabular}
    \end{center}
    \caption{各数字の画像におけるテストデータに対する識別精度(上:画像縮小なし,下:画像縮小あり)}
\end{table}
実験には各数字画像の20\%をトレーニングデータに使用し,残りをテストデータとした.\par
この表から分かる通り,一部の数字画像において識別精度が著しく下がる結果となった.
識別精度の下がった数字画像においては,画像縮小を行わなかった場合の識別精度においても低かったものが下がっており,比較的識別の難しい数字画像が縮小の影響を受けることで,識別することができなくなっていることが分かる.
この結果を受けて,実験では画像縮小を行わずに進めることとした.

\subsection*{実験結果}
最終的に識別結果の多数決でMNISTの数字画像を識別した結果は以下の通りである.
\begin{table}[H]
    \begin{center}
        \begin{tabular}{cc}
            \hline
            Target & Accuracy\\
            \hline \hline
            0 & 99.0\%\\
            1 & 46.0\%\\
            2 & 76.0\%\\
            3 & 95.0\%\\
            4 & 92.0\%\\
            5 & 42.0\%\\
            6 & 95.0\%\\
            7 & 82.0\%\\
            8 & 96.0\%\\
            9 & 75.0\%\\
            All & 79.8\%\\
            \hline
        \end{tabular}
        \caption{各数字画像における識別精度}
        \label{result}
    \end{center}
\end{table}
実験では学習で使われていない各数字画像100枚ずつ,計1000枚を用いて識別精度を測った.

\subsection*{考察}
表\ref{result}から分かるように,最終的な識別精度としては79.8\%となったが,各数字ごとに着目すると,識別精度は数字ごとにばらつきがあることが分かる.
特に"1"と"5"における識別精度が低くなっている.
2クラス分類の時点ではその傾向を読み取ることができなかったため,各2クラス識別器の識別精度とその識別結果の多数決による識別精度には必ずしも関係性があるわけではないと考えられる.