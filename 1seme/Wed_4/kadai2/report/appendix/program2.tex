\section{プログラム2}
\begin{lstlisting}[caption=課題1におけるSVM学習用プログラム]
    import cvxopt
    import matplotlib.pyplot as plt
    import numpy as np
    import os
    from sklearn.model_selection import train_test_split
    import MakeData
    os.chdir(os.path.dirname(os.path.abspath(__file__)))


    # データを学習できる形に変換
    def data_exchange(data1, data2):
        data = data1 + data2        # データを1つにまとめる
        label = label1 + label2     # ラベルを1つにまとめる
        data = np.array(data)       # データ配列をnumpy型に変換
        label = np.array(label)     # ラベル配列をnumpy型に変換
        train_data, test_data, train_label, test_label = train_test_split(data, label, test_size=0.3)       # トレーニング用とテスト用に分ける

        return train_data, test_data, train_label, test_label


    # SVM
    class SVM:
        def __init__(self, data, label):
            self.data = data
            self.label = label
        

        # 線形カーネル
        def kernel(self, x, y):
            return np.dot(x, y)


        # ラグランジュ乗数を二次計画法で求める
        def Lagrange(self, n):
            K = np.zeros((n, n))
            for i in range(n):
                for j in range(n):
                    K[i, j] = self.label[i] * self.label[j] * np.dot(self.data[i], self.data[j])
            Q = cvxopt.matrix(K)
            p = cvxopt.matrix(-np.ones(n))      # -1がn個の列ベクトル
            G = cvxopt.matrix(np.diag([-1.0]*n))        # 対角成分が-1の(n × n)行列
            h = cvxopt.matrix(np.zeros(n))      # 0がn個の列ベクトル
            A = cvxopt.matrix(self.label, (1,n))     # N個の教師信号が要素の行ベクトル(1 × n)
            b = cvxopt.matrix(0.0)      # 定数0.0
            solution = cvxopt.solvers.qp(Q, p, G, h, A, b)       # 二次計画法でラグランジュ乗数alphaを求める

            alpha = np.array(solution['x']).flatten()

            return alpha


        # サポートベクトルを抽出
        def support_vector(self, alpha):
            S = []
            for i in range(len(alpha)):
                if alpha[i] >= 0.00001:
                    S.append(i)
            
            return S


        # wを計算
        def w_cal(self, S, alpha):
            w = np.zeros(2)
            for n in S:
                w += alpha[n] * self.label[n] * self.data[n]
            
            return w

        
        # bを計算
        def b_cal(self, S, alpha):
            _sum = 0
            for n in S:
                tmp = 0
                for m in S:
                    tmp += alpha[m] * self.label[m] * self.kernel(self.data[n], self.data[m])
                _sum += (self.label[n] - tmp)
            b = _sum / len(S)

            return b


        def main(self):
            alpha = self.Lagrange(len(self.data))        # ラグランジュ乗数
            S = self.support_vector(alpha)       # サポートベクトル
            w = self.w_cal(S, alpha)     # w
            b = self.b_cal(S, alpha)        # b

            return w, b, S


    # ラベルごとにデータを分割
    def data_split(data, label):
        cls1 = []
        cls2 = []
        for i in range(len(data)):
            if label[i] == 1:
                cls1.append(data[i])
            elif label[i] == -1:
                cls2.append(data[i])
        
        return cls1, cls2


    def f(x, w, b):
        return np.dot(w, x) + b


    def f_plot(x1, w, b):
        return -(w[0] / w[1]) * x1 - (b / w[1])


    # 精度を計算
    def accuracy(cls1, cls2, w, b):
        num = len(cls1) + len(cls2)
        c = 0
        for i in cls1:
            if f(i, w, b) >= 0:
                c += 1
        for i in cls2:
            if f(i, w, b) < 0:
                c += 1

        acc = c / num

        return acc


    # 結果を描画
    class Draw:
        def __init__(self, data, cls1, cls2, x_min, x_max, y_min, y_max, w, b, S, acc, check):
            self.data = data
            self.cls1 = cls1
            self.cls2 = cls2
            self.x_min = x_min
            self.x_max = x_max
            self.y_min = y_min
            self.y_max = y_max
            self.w = w
            self.b = b
            self.S = S
            self.acc = acc
            self.check = check


        # 結果を描画
        def main(self):
            # 訓練データを描画
            x1, x2 = np.array(self.cls1).transpose()
            plt.plot(x1, x2, 'rx')    
            x1, x2 = np.array(self.cls2).transpose()
            plt.plot(x1, x2, 'bx')

            # サポートベクトルを描画
            if self.check == 'train':
                for n in self.S:
                    plt.scatter(self.data[n,0], self.data[n,1], s=80, c='c', marker='o')
            
            # 識別境界を描画
            x1 = np.linspace(self.x_min, self.x_max, 1000)
            x2 = [f_plot(x, self.w, self.b) for x in x1]
            plt.plot(x1, x2, 'g-')

            plt.xlim(self.x_min, self.x_max)
            plt.ylim(self.y_min, self.y_max)
            plt.text(-25, -25, 'Accuracy : {}%'.format(round(self.acc*100, 2)))
            plt.show()


    if __name__ == '__main__':
        N1 = 100     # クラス1のデータ数
        N2 = 100     # クラス2のデータ数

        # データを作成
        data1, label1 = MakeData.gauss(5, 2.5, 5, 2.5, 1.0, N1)       # クラス1のデータとラベル
        data2, label2 = MakeData.gauss(-5, 2.5, -5, 2.5, -1.0, N2)        # クラス2のデータとラベル
        train_data, test_data, train_label, test_label = data_exchange(data1, data2)

        # SVM
        svm = SVM(train_data, train_label)
        w, b, S = svm.main()

        # トレーニングデータ、テストデータをラベル別に分割
        cls1_train, cls2_train = data_split(train_data, train_label)
        cls1_test, cls2_test = data_split(test_data, test_label)

        # トレーニングデータ、テストデータのそれぞれに対して精度を計算
        acc_train = accuracy(cls1_train, cls2_train, w, b)
        acc_test = accuracy(cls1_test, cls2_test, w, b)
        print('Accuracy for training data : {}'.format(acc_train))
        print('Accuracy for test data : {}'.format(acc_test))
        
        x_min = -30      # xの最小値(描画範囲)
        x_max = 30      # xの最大値(描画範囲)
        y_min = -30      # yの最小値(描画範囲)
        y_max = 30      # yの最大値(描画範囲)

        # トレーニングデータとテストデータに対する結果を描画
        draw_train = Draw(train_data, cls1_train, cls2_train, x_min, x_max, y_min, y_max, w, b, S, acc_train, check='train')
        draw_test = Draw(train_data, cls1_test, cls2_test, x_min, x_max, y_min, y_max, w, b, S, acc_test, check='test')
        draw_train.main()
        draw_test.main()
\end{lstlisting}