\section{プログラム3}
\begin{lstlisting}[caption=課題2のMNISTの数字画像識別におけるSVM学習用プログラム]
    import cv2
    import cvxopt
    import numpy as np
    import os
    import pandas as pd
    import time
    from sklearn.model_selection import train_test_split
    os.chdir(os.path.dirname(os.path.abspath(__file__)))


    # データセットの作成
    class DATASET:
        def __init__(self, num1, num2, num3):
            self.cls = num1
            self.num1 = num2
            self.num2 = num3


        def one_vs_the_rest(self):
            data = []
            label = []
            print('Loading data (positive:{})...'.format(self.cls), end='')
            for num in range(0, 10):
                for root, dirs, files in os.walk('train_img/{}'.format(num)):
                    for f in files:
                        '''
                        # リサイズによる次元削減
                        img = cv2.imread('{}/{}'.format(root, f), 0)
                        dst = cv2.resize(img, dsize=None, fx=0.5, fy=0.5)
                        dst = dst.reshape(-1)
                        data.append(dst)
                        '''
                        data.append(cv2.imread('{}/{}'.format(root, f), 0).reshape(-1))     # 1次元配列に変換してdataに貯めていく
                        if num == self.cls:
                            label.append(1.0)
                        else:
                            label.append(-1.0)
            print('Finish!')
            
            data = np.array(data)
            label = np.array(label)

            data = data / 255.0

            train_data, test_data, train_label, test_label = train_test_split(data, label, test_size=0.9)
            
            return train_data, test_data, train_label, test_label


        def one_vs_one(self):
            data = []
            label = []
            print('Loading data (\"pos\"_\"neg\":{}_{})...'.format(self.num1, self.num2), end='')
            for root, dirs, files in os.walk('train_img/{}'.format(self.num1)):
                for f in files:
                    '''
                    # リサイズによる次元削減
                    img = cv2.imread('{}/{}'.format(root, f), 0)
                    dst = cv2.resize(img, dsize=None, fx=0.5, fy=0.5)
                    dst = dst.reshape(-1)
                    data.append(dst)
                    '''
                    data.append(cv2.imread('{}/{}'.format(root, f), 0).reshape(-1))     # 1次元配列に変換してdataに貯めていく
                    label.append(1.0)
            for root, dirs, files in os.walk('train_img/{}'.format(self.num2)):
                for f in files:
                    '''
                    # リサイズによる次元削減
                    img = cv2.imread('{}/{}'.format(root, f), 0)
                    dst = cv2.resize(img, dsize=None, fx=0.5, fy=0.5)
                    dst = dst.reshape(-1)
                    data.append(dst)
                    '''
                    data.append(cv2.imread('{}/{}'.format(root, f), 0).reshape(-1))     # 1次元配列に変換してdataに貯めていく
                    label.append(-1.0)
            print('Finish!')
            
            data = np.array(data)
            label = np.array(label)

            data = data / 255.0

            train_data, test_data, train_label, test_label = train_test_split(data, label, test_size=0.8)
            
            return train_data, test_data, train_label, test_label


    # SVM
    class SVM:
        def __init__(self, data, label):
            self.data = data
            self.label = label
        

        # 線形カーネル
        def kernel(self, x, y):
            return np.dot(x, y)


        # ラグランジュ乗数を二次計画法で求める
        def Lagrange(self, n):
            K = np.zeros((n, n))
            for i in range(n):
                for j in range(n):
                    K[i, j] = self.label[i] * self.label[j] * np.dot(self.data[i], self.data[j])
            Q = cvxopt.matrix(K)
            p = cvxopt.matrix(-np.ones(n))      # -1がn個の列ベクトル
            G = cvxopt.matrix(np.diag([-1.0]*n))        # 対角成分が-1の(n × n)行列
            h = cvxopt.matrix(np.zeros(n))      # 0がn個の列ベクトル
            A = cvxopt.matrix(self.label, (1,n))     # N個の教師信号が要素の行ベクトル(1 × n)
            b = cvxopt.matrix(0.0)      # 定数0.0
            solution = cvxopt.solvers.qp(Q, p, G, h, A, b)       # 二次計画法でラグランジュ乗数alphaを求める

            alpha = np.array(solution['x']).flatten()

            return alpha


        # サポートベクトルを抽出
        def support_vector(self, alpha):
            S = []
            for i in range(len(alpha)):
                if alpha[i] >= 0.00001:
                    S.append(i)
            
            return S


        # wを計算
        def w_cal(self, S, alpha):
            w = np.zeros(784)
            for n in S:
                w += alpha[n] * self.label[n] * self.data[n]
            
            return w

        
        # bを計算
        def b_cal(self, S, alpha):
            _sum = 0
            for n in S:
                tmp = 0
                for m in S:
                    tmp += alpha[m] * self.label[m] * self.kernel(self.data[n], self.data[m])
                _sum += (self.label[n] - tmp)
            b = _sum / len(S)

            return b


        def main(self):
            alpha = self.Lagrange(len(self.data))        # ラグランジュ乗数
            S = self.support_vector(alpha)       # サポートベクトル
            w = self.w_cal(S, alpha)     # w
            b = self.b_cal(S, alpha)        # b

            return w, b, S


    # ラベルごとにデータを分割
    def data_split(data, label):
        cls1 = []
        cls2 = []
        for i in range(len(data)):
            if label[i] == 1:
                cls1.append(data[i])
            elif label[i] == -1:
                cls2.append(data[i])
        
        return cls1, cls2


    def f(x, w, b):
        return np.dot(w, x) + b


    # 精度を計算
    def accuracy(cls1, cls2, w, b):
        num = len(cls1) + len(cls2)
        c1 = 0
        c2 = 0
        for i in cls1:
            if f(i, w, b) >= 0:
                c1 += 1
            elif f(i, w, b) < 0:
                c2 += 1
        for i in cls2:
            if f(i, w, b) < 0:
                c1 += 1
            elif f(i, w, b) >= 0:
                c2 += 1
        if c1 > c2:
            acc = c1 / num
        elif c1 < c2:
            acc = c2 / num

        return acc


    # 保存
    class SAVE:
        def __init__(self, output_w, output_b, output_acc_train, output_acc_test, processing_time):
            self.output_w = output_w
            self.output_b = output_b
            self.output_acc_train = output_acc_train
            self.output_acc_test = output_acc_test
            self.processing_time = processing_time

        
        def one_vs_the_rest(self):
            # 学習結果のパラメータを保存
            df = pd.DataFrame(self.output_w, index=['0', '1', '2', '3', '4', '5', '6', '7', '8', '9'])
            df['b'] = self.output_b

            dirname = 'classifier'
            if not os.path.exists('{}'.format(dirname)):
                os.mkdir('{}'.format(dirname))

            file_num = 1
            while 1:
                if not os.path.exists('{}/one_versus_the_rest/SVM{}.csv'.format(dirname, file_num)):
                    df.to_csv('{}/one_versus_the_rest/SVM{}.csv'.format(dirname, file_num))
                    print('Save classifier as \"SVM{}.csv\"'.format(file_num))
                    break
                else:
                    file_num += 1

            # 各学習の精度と学習時間を保存
            df = pd.DataFrame({'acc_for_train':self.output_acc_train, 'acc_for_test':self.output_acc_test, 'processing_time':self.processing_time}, index=['0', '1', '2', '3', '4', '5', '6', '7', '8', '9'])
            
            dirname = 'result'
            if not os.path.exists('{}'.format(dirname)):
                os.mkdir('{}'.format(dirname))

            file_num = 1
            while 1:
                if not os.path.exists('{}/one_versus_the_rest/result{}.csv'.format(dirname, file_num)):
                    df.to_csv('{}/one_versus_the_rest/result{}.csv'.format(dirname, file_num))
                    print('Save accuracy and processing time as \"result{}.csv\"'.format(file_num))
                    break
                else:
                    file_num += 1


        def one_vs_one(self):
            # 学習結果のパラメータを保存
            id = []
            for i in range(0, 10):
                for j in range(i+1, 10):
                    id.append('{}_{}'.format(i, j))

            df = pd.DataFrame(self.output_w, index=id)
            df['b'] = self.output_b

            dirname = 'classifier'
            if not os.path.exists('{}'.format(dirname)):
                os.mkdir('{}'.format(dirname))

            file_num = 1
            while 1:
                if not os.path.exists('{}/one_versus_one/SVM{}.csv'.format(dirname, file_num)):
                    df.to_csv('{}/one_versus_one/SVM{}.csv'.format(dirname, file_num))
                    print('Save classifier as \"SVM{}.csv\"'.format(file_num))
                    break
                else:
                    file_num += 1

            # 各学習の精度と学習時間を保存
            df = pd.DataFrame({'acc_for_train':self.output_acc_train, 'acc_for_test':self.output_acc_test, 'processing_time':self.processing_time}, index=id)
            
            dirname = 'result'
            if not os.path.exists('{}'.format(dirname)):
                os.mkdir('{}'.format(dirname))

            file_num = 1
            while 1:
                if not os.path.exists('{}/one_versus_one/result{}.csv'.format(dirname, file_num)):
                    df.to_csv('{}/one_versus_one/result{}.csv'.format(dirname, file_num))
                    print('Save accuracy and processing time as \"result{}.csv\"'.format(file_num))
                    break
                else:
                    file_num += 1


    if __name__ == '__main__':
        while 1:
            svm_type = int(input('\"one_versus_the_rest(0)\" or \"one_versus_one(1)\"? : '))
            if svm_type == 0 or svm_type == 1:
                break
            else:
                print('Error. Please, input \"0\" or \"1\"')
        output_w = []
        output_b = []
        output_acc_train = []
        output_acc_test = []
        processing_time = []

        if svm_type == 0:
            for num in range(0, 10):
                print('==============================')

                dataset_make = DATASET(num, None, None)
                train_data, test_data, train_label, test_label = dataset_make.one_vs_the_rest()

                # SVM
                print('Start learning')
                svm = SVM(train_data, train_label)
                start = time.time()
                w, b, S = svm.main()
                elapsed_time = time.time() - start
                processing_time.append(elapsed_time)
                print('Finish!')

                output_w.append(w)
                output_b.append(b)

                # トレーニングデータ、テストデータをラベル別に分割
                cls1_train, cls2_train = data_split(train_data, train_label)
                cls1_test, cls2_test = data_split(test_data, test_label)

                # トレーニングデータ、テストデータのそれぞれに対して精度を計算
                acc_train = accuracy(cls1_train, cls2_train, w, b)
                output_acc_train.append(acc_train)
                acc_test = accuracy(cls1_test, cls2_test, w, b)
                output_acc_test.append(acc_test)
                print('Accuracy for training data (classify \"{}\") : {}'.format(num, acc_train))
                print('Accuracy for test data (classify \"{}\") : {}'.format(num, acc_test))

            print('==============================')

        elif svm_type == 1:
            for num1 in range(0, 10):
                for num2 in range(num1+1, 10):
                    print('==============================')

                    dataset_make = DATASET(None, num1, num2)
                    train_data, test_data, train_label, test_label = dataset_make.one_vs_one()

                    # SVM
                    print('Start learning')
                    svm = SVM(train_data, train_label)
                    start = time.time()
                    w, b, S = svm.main()
                    elapsed_time = time.time() - start
                    processing_time.append(elapsed_time)
                    print('Finish!')

                    output_w.append(w)
                    output_b.append(b)

                    # トレーニングデータ、テストデータをラベル別に分割
                    cls1_train, cls2_train = data_split(train_data, train_label)
                    cls1_test, cls2_test = data_split(test_data, test_label)

                    # トレーニングデータ、テストデータのそれぞれに対して精度を計算
                    acc_train = accuracy(cls1_train, cls2_train, w, b)
                    output_acc_train.append(acc_train)
                    acc_test = accuracy(cls1_test, cls2_test, w, b)
                    output_acc_test.append(acc_test)
                    print('Accuracy for training data (classify \"{}_{}\") : {}'.format(num1, num2, acc_train))
                    print('Accuracy for test data (classify \"{}_{}\") : {}'.format(num1, num2, acc_test))

            print('==============================')

        # 学習結果のパラメータ、各学習の精度と学習時間を保存
        save = SAVE(output_w, output_b, output_acc_train, output_acc_test, processing_time)
        if svm_type == 0:
            save.one_vs_the_rest()
        elif svm_type == 1:
            save.one_vs_one()
\end{lstlisting}