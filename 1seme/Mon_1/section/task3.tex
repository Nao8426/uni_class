\section*{問題3}
第5世代セルラシステムの利用分野は,第4世代セルラシステムまでと比較してどのような点が大きく異なるのかを説明せよ.
また,その違いを生んだ理由について説明せよ.

\subsubsection*{レポート}
過去の変遷をたどってみると,第2世代セルラシステムの登場により,携帯電話によるメールやネットの利用を可能にした.
これは基本的に音声の通信のみであった第1世代セルラシステムに対して,大きな進歩であったと言える.
しかし,伝送速度も遅く,データ量の多い情報を扱うには限界があった.
その後登場した第3世代,第4世代セルラシステムにより,高速・大容量のデータ通信が可能となり,携帯電話は情報通信サービスとして本格化することとなった.
このように第4世代セルラシステムまでは,主に携帯電話サービスという分野で利用されてきた.\par
そして今,第5世代セルラシステムが注目を浴びている.
今までのセルラシステムよりも高速化・大容量化しているのは確かだが,それが活きる場面は携帯電話サービスのみに限らない.
今の時代は,様々なデバイスがネットワークへの接続機能を持っている,「IoT時代」である.
携帯電話の普及も相まって,あらゆるデバイスにおける接続端末数は膨大なものになると予想される.
このことから,第4世代セルラシステムでは,同時に接続する端末が増え過ぎることによって,通信品質の低下という問題が生じる可能性がある.
そこで,この問題を解決するために第5世代セルラシステムが開発されている.\par
第5世代セルラシステムがこのような問題を解決することを可能とする背景には,主に「高周波数帯の利用」と「超多素子アンテナ技術の利用」がある.
一般的に電波の周波数が高くなると伝送できる情報量が増える代わりに,障害物に弱くなる性質がある.
それを回避し,通信ネットワークとして実用化する技術が研究されており,その技術を用いることで,伝送可能な情報量の増加を実現する.
また,MIMOと呼ばれる超多素子アンテナを用いた通信によって,使用する周波数帯域を増やすことなく,通信の高速化,通信品質の向上が可能となる.
このように技術が発達したことによって,第5世代セルラシステムはIoT時代においても十分通用するシステムとして期待されている.
