\section*{問題2}
セルラ通信において位置登録はどのようなメカニズムで行われるのかを説明せよ.

\subsubsection*{レポート}
全ての無線基地局は,それが所属している地域を表す信号を常に送信しており,移動局はそれをもとに自分がいる地域を認識している.
新しい地域に移動したことが検出された場合,移動局はその地域をホームメモリ局に報告する.
その際,ホームメモリ局は移動局の位置をセル単位で把握することができるが,セル単位で位置管理を行うと,移動局はセルを移動するたびに地域更新のために位置登録信号を送信しなければならず,通信量が膨大になる.
そこで,実際にはそこまで細かく管理せず,複数のセルをまとめて位置登録エリアとし,その位置登録エリア単位で管理する.\par
しかし,このままでは移動局が位置登録エリア内のどのセルにいるのかが不明なため,着信接続の際には以下のような工夫が必要となる.
まず,移動局Aの電話番号がダイヤルされると,ダイヤルの移動網識別番号によって,ゲートウェイ交換局に接続される.
次に,ゲートウェイ交換局はホームメモリ局に接続することで,移動局Aのいる位置登録エリアを特定し,その地域のローカル交換局へ接続する.
交換局はローカル交換局はその地域の全無線基地局に対して一斉呼出し要求を行い,移動局Aは呼出し信号を受信すると応答信号を送信する.
最後に,ローカル交換局は応答信号を受信すると,その地域の全無線基地局に対して一斉呼出しの中止を指示する.
通信をするためには,この後正規ユーザであることの確認などの工程が必要であるが,位置登録と実際に用いる際の工夫に関してはここまで示したような流れで行われる.

% オーバーレイ位置登録
% セル方式による移動通信システムでは一般に複数のセルからなる登録エリア単位で位置登録を行うが,その境界では受信レベル変動にともなって頻繁に位置登録の切り替えが生じてしまう.
% オーバレイ位置登録は,移動機の初期位置登録点を中心として,各移動機ごとに個別エリアを設定する方式で,そのエリアをはみ出した場合は,また新たな位置登録点を中心としたエリア設定するため,位置登録トラヒックの大幅な減少をはかることができる.
