\section{Work1}
\subsection*{問題}
何らかの識別問題または回帰問題を設定し,それを機械学習により解く.
さらに,評価データとして,学習データをそのまま使用,学習データとは異なるデータを使用,の2つの場合の性能を比較する.

\subsection{概要}
MNISTの0~9の数字が描かれた画像に対して,機械学習によって描かれている数字を分類する分類器を作成し,その精度を測った.
機械学習の手法にはSVM(Support Vector Machine)を用いた.
以下ではその結果と考察を述べる.

\subsection{結果}
分類の精度を以下に示す.
\begin{table}[H]
    \begin{center}
        \begin{tabular}{cc}
            \hline
            Data & Accuracy\\
            \hline \hline
            Training data & 99.0\%\\
            Test data & 97.9\%\\
            \hline
        \end{tabular}
    \end{center}
\end{table}

\subsection{考察}
実際に精度を測る際に、学習データをそのまま用いて測ることをオープンテスト(open test),学習に使わずに用意しておいたテスト用のデータを用いて測ることをクローズドテスト(closed test)と言う.
今回の分類に関しても両者で精度を測り,結果として,オープンテストでは99.0\%,クローズドテストでは97.9\%という結果が得られた.
比較してみると,オープンテストの方がクローズドテストよりもわずかに高い精度が達成された.
学習の際には学習用のデータをうまく分類できるように学習するため,学習用のデータに対しては精度が高く,それと比較して,未知のデータであるテスト用のデータに対しては低くなるのは妥当な結果であると言えるだろう.\par
今回のように,データを学習用とテスト用のデータに分けて,それぞれのデータを用いて学習とテストを行うことを交差検証と言う.
これは,学習用のデータを分類することに適合し過ぎるあまり,未知のデータに対する分類がうまくいかない(過学習)といった問題を防ぐことが期待される.
