\section*{問題1}
正例包絡では,訓練データが全て正しく判別できるように境界を決めるため,訓練データに対する精度は基本的に100\%となる.
よって,正例包絡において訓練データに対する性能が検証データに対する性能より悪くなることはない.

しかし,一般的な機械学習では必ずしも訓練データ全てを正しく判別できるように学習するわけではなく,外れ値を無視するなどの計算が行われる.
よって一般的な機械学習では,訓練データに対する性能が検証データに対する性能より悪くなるとは言い切れない.