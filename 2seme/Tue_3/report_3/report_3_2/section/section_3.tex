\section*{問題3}
\noindent
まず,$B_i\subset A_i\Rightarrow \bm{x}_j\notin B_i,\ {\rm for\ all}\ j=1,\ldots,n$について証明する.\\
$B_i\subset A_i$のとき,$\bm{x}_j\in B_i$となる$\bm{x}_j$が存在すると仮定する.\\
上式から$\bm{x}_j\subset A_i$となる.\\
また,$\bm{x}_j\subset \hat{C}_{\rm fit},\ {\rm for\ all}\ j=1,\ldots,n$であるから$A_i$と$\hat{C}_{\rm fit}$は一部重複することになる.\\
しかし定義より,$A_i$と$\hat{C}_{\rm fit}$は互いに排反であるのでこれに矛盾する.\\
よって仮定が間違っていることから,$B_i\subset A_i\Rightarrow \bm{x}_j\notin B_i,\ {\rm for\ all}\ j=1,\ldots,n$が示された.\\
次に,$B_i\subset A_i\Leftarrow \bm{x}_j\notin B_i,\ {\rm for\ all}\ j=1,\ldots,n$について証明する.\\
$\bm{x}_j\notin B_i,\ {\rm for\ all}\ j=1,\ldots,n$より,$B_i$と$\hat{C}_{\rm fit}$は排反である.\\
さらに$B_i\subset C^*$より,$B_i\subset A_i$となる.\\
よって,$B_i\subset A_i\Leftarrow \bm{x}_j\notin B_i,\ {\rm for\ all}\ j=1,\ldots,n$が示された.\\
ゆえに,$B_i\subset A_i\Leftrightarrow \bm{x}_j\notin B_i,\ {\rm for\ all}\ j=1,\ldots,n$が示された.