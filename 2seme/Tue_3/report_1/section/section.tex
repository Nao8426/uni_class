\section*{問題1}
数え上げ測度は単調性を有する.

数え上げ測度は,任意の集合に対してその元の個数によって定義される測度のことである.
$A \subseteq B$のとき,集合$A$に含まれる元の数は必ず集合$B$に含まれる元の数よりも小さくなる($|A| \leqq |B|$).
よって$A \subseteq B$ならば$f(A) \leqq f(B)$であることが分かる.
ゆえに数え上げ測度は単調性を満たす.

\section*{問題2}
私が考えたアプリケーションは,決定木によって目当ての店を探索するというものです.
その人がいる場所や日時,またその人が求めている物の種類や値段などから,どの店に行けばよいのかを決定木で決めます.

まずはその人がいる場所や店に行きたい日時などから,店の分類を行います.
次に種類ですが,その人が求めているもの家具なのか服なのか,あるいは食べ物なのかといった情報から分類を行います.
同じように,その値段の価格帯情報からさらに店を絞ります.

分類に決定木を用いることで,その探索結果に至った経緯が分かりやすくなり,その店にどのような需要があって客が来ているのかということが分かります.
店側もその情報を利用することで,顧客のニーズを容易に把握することが可能となります.