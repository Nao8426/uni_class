\section*{\huge 自動運転における深層学習 \hspace{15mm} \large 濱崎 直紀(学籍番号:28G19096)}
近年,機械学習は幅広い場面で利用され,その有用性が示されている.
そのようなものの一つに自動運転技術がある.
車の運転の過程は主に認知・判断・操作に分けられる.
まず,運転手は周囲の状況(歩行者の有無や信号の確認,標識の認識など)を認知し,それに対して適切な行動が何であるかを判断し,それに従って実際に操作を行うのである.
一般的には,この認知・判断・操作の過程を人が行うわけであるが,それを人間の代わりにシステムが行うのが自動運転である.

この過程の認知において機械学習は欠かせないものとなっている.
自動運転では,周りの状況を把握するため,車にセンサーが取り付けられており,これが人で言うところの目の働きをしている.
しかし,見るという行為は言わばデータを入手するだけであり,そのデータから情報を得る必要がある.
人で言うと脳の働きであるこの行程に深層学習が活かされている.

ここで深層学習とは,大量のデータを用いてそのデータに含まれる特徴を段階的により深く学習する機械学習のことを指す.
深層学習のアルゴリズムはパーセプトロンというものがベースとなっている.
パーセプトロンは人の脳の神経細胞をモデル化したものである.
基本的なパーセプトロンの出力$y$は,入力$\bm{x}$,それぞれに対する重み$\bm{w}$,バイアス$b$,活性化関数$f$を用いて$y=f(\bm{wx}+b)$で表され,この重みを更新していくことで学習が行われる.
深層学習はこのパーセプトロンを多数組み合わせたニューラルネットワークを多層化することで実現されている.
従来の手法よりも複雑な解析が可能であることから,機械学習において現在主流になっている学習方法である.

では具体的にどのように深層学習が用いられているのか.
センサーから取得されたデータはただの映像であり,その中から歩行者,車,信号,標識あるいは道路の白線といったものをシステムが認識する必要がある.
しかし,車と一言で言ってもその見た目(色や形)には膨大な種類がある.
車だけではなく,もちろん歩行者,信号,標識など,様々な物体に様々な種類があるので,それら全てをデータとして登録しておくことは困難である.
さらには,周囲の環境(明るさなど)によってその見た目にも微妙な変化が加わってしまう問題もある.
このような非常に困難な認識を行う際に役立つのが深層学習である.
様々な角度から見た車の外見,光の当たり具合で見え方が変わる色などの条件も含めて,コンピュータが深層学習を用いて学習することで,人間の脳のように瞬時に認識した物体を解析し,たとえ識別対象に該当する物体が保持しているデータの中に存在していなくとも,その外見から読み取った特徴によって正しく識別することが可能となる.

現状,完全に自動運転とまではいかないものの,部分的に自動化を行っている車は既に実用化されており,自動運転車においても試乗という形で実現はされている.
機械学習技術も発展していることを考えると,一般道で自動運転車を見る日もそう遠くないだろう.