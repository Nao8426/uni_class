\section*{課題1}
「カーツワイルの未来予測」および「人体 神秘のネットワーク」,「石黒浩教授の最後の授業」などから,ロボットと共生する未来社会の実現イメージを提案し,自らの期待や不安,あるべき方向について意見を述べよ.

\subsection*{(回答)}
私がイメージしたロボットと共生する未来社会は,人が生活する環境(家,会社などの建物)自体をロボットにするという形の共生社会です.
家が,効率的な間取りやその時の目的に応じた部屋の形式に自律的に変形したり,清潔に保つために勝手に掃除が行われるなど,様々な想像ができます.
建物自体がロボットになれば,老朽化などのメンテナンスをロボット自体が管理するなど,効率化以外にも安全な生活という点にも期待ができます.
このような技術の発展は,高齢化が進んでいる今の日本の社会では特に重要であり,非常に役立つものだと思います.
しかし,生活の環境というものは生きる上でとても重要であり,そこをロボットに完全に任せるのは簡単なことではないと思います.
例えば,メンテナンスにおいてロボットが何かしらのエラーで正常に点検ができておらず,それに気づかずに放置されると,倒壊の恐れなどが出てきます.
このような問題は,どの活用方法でも問題になるものと思いますが,生活環境という非常に身近な場合では,問題が生命の危機にまで発展する恐れがあります.
もちろん,人であろうとロボットであろうとミスやエラーは付き物なので,大きな問題では無いようにも見えますが,ロボットの場合はその責任を誰が負担するのかなど,別の問題にも発展しかねません.
技術の発展ももちろんですが,制度をしっかりと整備して,実用上の問題を慎重に精査することが大切だと思います.
