\section*{課題2}
サイバネティック・アバターの2050年と2100年の未来社会の実現イメージを提案し,自らの期待や不安,あるべき方向について500~1000文字程度で意見を述べよ.

\subsection*{(回答)}
2050年には,自分の体の動き,あるいは想像した動きが遠隔のアバターに反映され,重労働や通常ではできない動きを再現するときに活用できるような技術になっていると思います.
さらに技術が発展した2100年には,複数のアバター操作が可能になり,あるところでは仕事をし,あるところでは別の作業をするなど,一人当たりの労働力が何十倍にも跳ね上がるような技術が期待されます.
あらゆることを一斉にすることで,疲弊してしまうという懸念点もありますが,感覚まで同期できるのであれば,リラックスをするアバターも作ることで,息抜きも同時に行えるのかもしれません.

しかし,あまりにもサイバー上の世界が現実に近づいてしまうと,現実との区別がつかなくなる恐れがあると思います.
現実世界ではできもしないことを試みてしまい失敗し,時にはそれが命の危険を伴うかもしれません.
また,サイバネティック・アバターに慣れ切ってしまい,依存するようになると,現実での身体能力が低下してしまう可能性もあります.
これらのことがどれだけ問題となるのかは分かりませんが,少なくとも考慮すべき問題ではあると思いました.

とはいえ,人が自分の能力の限界に捕われることなく,自由自在に活動できる技術は,とても魅力的であり,人々が生活する上で非常に有用な技術だと思いました.
