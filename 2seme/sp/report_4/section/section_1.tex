\section*{テーマ}
これからの世界のために考えていること

\subsection*{(回答)}
近年の世界情勢はコロナによって大きく変わりました.
ソーシャルディスタンスを保つことが重要視されるようになり,人と人との繋がりが希薄になるなど,様々なことが懸念されています.
しかし近年,通信技術の発達によってオンライン上で作業をしたりコミュニケーションをとったりする機会が増えているということを考えると,これは生活様式の転換としてはチャンスであると思います.

そこで私は通信技術の発達によって,人々のコミュニケーションを支援したいと思っていました.
現状として,遠隔のコミュニケーションは基本的には画面越しの映像や音声などのやり取りになっています.
それをさらにリアルに近いものへと近づける,実際に会って話すのと変わらないリアリティを実現することで,遠隔コミュニケーションの質は必ず上がると思います.
例えば,ホログラムによって相手の姿や動き,環境などを再現することで,より細かな情報まで鮮明に伝えることが可能になるなどが考えられます.
現在,VRやARなどの技術が実際に使用されてることを考えると,このような技術も遠い未来の話ではないと思います.

おそらく,コロナの騒ぎが収まった後も,通信技術は発達し,遠隔での作業やコミュニケーションの機会はますます増えていくことが予想されます.
それに対し,不安の声も度々聞こえますが,自分はむしろ技術の発展した社会に期待を抱いています.
