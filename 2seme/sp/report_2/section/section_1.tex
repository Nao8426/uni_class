\section*{今後の地域社会における情報通信産業の役割}
コロナの影響を受け,ここ半年ほどで世の中の情勢は大きく変わりました.
特に,人々の関わり方が大きく変わり,密を避け,ソーシャルディスタンスを保つことが重要視されるようになっています.
これまでも地域社会において情報通信産業は様々な形で利用されていますが,コロナ禍の今,さらに活用に期待がかかっていると思います.
あらゆる企業がテレワークという形をとり,今やそれが新たな働き方として受け入れられるという傾向も見られます.
普段の生活においても,遠く離れた人々がデバイスの映像,音声を通して,簡単に繋がることが可能になっています.
このように,情報通信産業は今の時代に非常にマッチしており,これから益々発展していくと考えています.

また,上記のような活用は自然災害対策においても大切であると思います.
日本は災害大国であり,地震,台風,大雨,洪水,火山噴火など,ありとあらゆる災害がこれからも多発するでしょう.
年々,災害対策も強化されており,被害を抑える試みにも情報通信技術が非常に大きな貢献をしていると思います.
しかし,被害を完全に無くすことは厳しく,大なり小なり被害を受けることは避けられないことです.
被害を受けた場合は,復旧・復興が必要になります.
その上で,人々が災害に対する情報を得たり,また人々同士が情報の共有を行い,協力していくことは非常に大切であり,そこに情報通信技術は必要不可欠なものだと思います.

このように,人と人を繋ぐ技術は様々な形で人々の生活を支援できるものだと思います.
これからも情報通信産業は様々な場面で活用されることと思いますが,人々の安心・安全を守るという点で非常に大きな役割を担っていると思います.
