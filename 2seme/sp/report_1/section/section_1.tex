\section*{課題1}
電力会社では様々な情報通信技術を最大限に活用して,電力の安定供給やお客さまへのサービスレベルを向上するために独自の情報通信システムを構築しています.
この中でも特に電力の安定供給に対する情報通信の役割について,本日の説明に基づき,知識を整理してまとめてください.

\subsection*{(回答)}
天候,気温,曜日,テレビ番組などの要因によって,電気の消費量(需要)は刻々と変化する.
しかし,この需要はコントロールすることができないため,需要に合わせて発電力(供給)をコントロールする必要がある.
この需要が供給を上回ると周波数が下がり,逆に供給が需要を上回ると周波数が上がるので,需要と供給のバランス調整において周波数は重要な役割を果たしている.
この周波数の細かい調整は人間の手で行っているわけではなく,システムによって調整されている.
その際に,現状の周波数を知るために各所に発電量と周波数を取り込む装置を置いておき,その装置によって測定されたデータをシステムに与えることで,周波数の調整が行われる.
ここに情報通信技術が活かされている.
災害などが起きた際に情報が得られなくなると困るため,耐災害性に優れた高品質な通信回線を構築することが重要となってくる.

このような受給調整以外にも,系統に事故が発生した際に,電力系統を安定に運用するために事故電流情報を瞬時に伝送する技術などにも情報通信技術が活用されている.

上記のことを踏まえると,情報通信技術が電力の安定供給を支えるシステムを構築する上で重要な役割を果たしていると考えられる.
