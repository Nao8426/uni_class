\section{Chapter 7 : SECURITY OF BIOMETRIC SYSTEM}
生体認証システムが満たす必要のある要件は主に3つ挙げられる.
1つは整合性である.
これは否認防止認証を保証する能力によって決まる.
否認防止とは,情報システムの利用や操作,データの送信などに対して,特定の人物が行ったことを後に証明できることを示す.
2つ目は可用性である.
これは正規ユーザーが保護されたサービスへのタイムリーで信頼性の高いアクセスを持っているかどうかによって決まる.
3つ目は機密性である.
これは保存された個人データが意図された目的のためだけに使用されるかということである.
上記の3つの要件全てが満たされていない場合,生体認証システムは失敗していると見なされる.
この失敗を引き起こす障害は種類に応じてDaniel-of-service(DoS),Instruction,Repudiation,Function creepに分けられる.

生体認証の信頼はこれらの脅威から保護する能力にかかっているが,絶対に安全で確実なシステムはなく,確保されるセキュリティのレベルはアプリケーションに依存する.

生体認証システムのセキュリティを分析する最初のステップは,様々な脅威エージェントと攻撃する脅威モデルを定義することである.
ここで脅威エージェントには2種類考えられる.
1つは他人受容,本人拒否やサンプル取得の失敗,登録の失敗などのようなシステムにおける障害である.
これらの障害は意図的な攻撃ではなく,生体認証システムの様々なモジュールが原因で発生する(ゼロエフォート攻撃).
もう1つは詐称者,攻撃者によって引き起こされる障害である.
詐称者は別の登録者になりすまそうとする個人を指し,攻撃者は生体認証システムの動作を妨害しようとする個人を指す.
ここでは,攻撃者が実行できる攻撃に焦点を当てる.

そのような攻撃として,まずインサイダー攻撃が挙げられる.
これは正規ユーザー自身がシステムを意図的に破壊する場合だけでなく,外部の敵がインサイダーの直接的または間接的な関与により生体認証システムを回避する場合も含む.
生体認証システムでは多くの段階で人間の関与が必要であり,これが悪用される恐れがある.
% 1つ目は「共謀」である.
% これは正規ユーザが外部の人間と協力して攻撃することを指す.
% 2つ目は「脅迫」である.
% これは正規ユーザが外部の人間から脅迫され,非積極的に攻撃することを指す.
% 3つ目は「過失」である.
% これは攻撃者が正規ユーザの過失を悪用することを指す.
% 4つ目は「登録詐欺」である.
% これは外部の人間が偽の資格情報とともに生体認証特性を生成することにより,システムに不法に登録することを指す.
% 5つ目は「例外の悪用」である.
% 生体認証システムには例外的な状況の処理を可能にするためにフォールバックメカニズムが用意されている.
% このフォールバックメカニズムの抜け穴を攻撃者が悪用することを指している.

もう1つの例としてインフラストラクチャ攻撃がある.
これは攻撃者が生体認証インフラストラクチャを操作することでセキュリティを侵害するものである.

% 7.3
% ここに何か必要
他の攻撃パターンとして,攻撃者が生体認証特性を提示して,システムに侵入しようとする試みがある.
これはユーザーインターフェースレベルでの攻撃と見なすことができ,次のような攻撃と対策が存在する.

% ここらへんはもう少し詳しくできる
% なりすましの区別(impersonation, spoofing)
まずは「なりすまし(impersonation)」である.
これは詐欺師自身が別の正規ユーザになりすましてシステムに侵入しようとする状況を指す.
これには生体認証システムの誤一致率(FMR)をできる限り低くし,時間枠内で認証における失敗回数を制限することにより対策が可能である.

次に「難読化」である.
これは生体認証システムによる検出を回避するために攻撃者が生体認証特性を変更しようとする状況を指す.
つまり,難読化は攻撃者が自分の身元を隠したい場合に使用される.
この攻撃の対策としては,ユーザ内変動に対してのシステムの堅牢性を改善することや,変更された部分を検出し,そのユーザーを二次検査にかけることなどが挙げられる.

% spoofingがよく分からない・・・
最後に「なりすまし(spoofing)」である.
これは他人から取得した生体認証特性から偽造した生体認証特性を用いて認証を試みる攻撃のことである.
これを検出することは,本人の生体認証特性と他のソースを介した生体認証特性を区別することであり,そのようなシステムの開発が必要である.

% 7.4
また,攻撃者が生体認証システムのモジュールを直接弱体化したり,モジュール間の通信を操作することによる攻撃も存在する.
これは,攻撃者が不正な変更を行うか,実装における障害を悪用することによって実現される.

不正な変更の例として「トロイの木馬」がある.
トロイの木馬は自身をモジュールの1つとして偽装することでモジュール間の通信に侵入し,攻撃者が希望する値を後続のモジュールへ渡すことが可能となる.
この攻撃の対策は,モジュール間の相互認証を使用して通信の双方向における信頼性が確立している生体認証システムを用いることである.
加えて,ソフトウェアの安全な実行を強制できるコード実行プラクティスまたは改ざん防止ハードウェアを使用することも有用である.

また,実際には無視できる例外入力などを用いた攻撃など,システムの抜け穴を悪用される恐れもある.
攻撃者はこれを実行するために少なくとも1つ以上のモジュールを経由する必要があるため,十分にテストされた生体認証アルゴリズムを使用することで,この悪用を防ぐことができる.

モジュール間の通信における攻撃には「中間者攻撃」「リプレイ攻撃」「ヒルクライミング攻撃」がある.
中間者攻撃はすでに通信中の2つの接続間に独立した接続を確立し,それらの間でメッセージを中継し盗聴を行う.
この対策としては,生体認証モジュール間の相互認証が挙げられる.
リプレイ攻撃は保護されていない転送中のデータを傍受し,不正に利用する攻撃である.
この対策として,その時限りのキーを生成するワンタイムセッションキーなどがある.
ヒルクライミング攻撃は,一致スコア情報を利用し,人工的に生成された特徴ベクトルからその応答の一致スコアを記録していき,より高いスコアがでた場合はその特徴ベクトルを保持するという反復を行うことで,設定された閾値を越えることを目的とした攻撃である.
時間枠内における失敗回数の制限などが主な対策となる.

最後に,生体認証テンプレートのデータベースへの攻撃が2種類考えられる.
1つは,テンプレートデータベースのハッキングまたは変更である.
これにより,不正なアクセスや正規ユーザーのアクセス拒否が可能になる.
このような脅威を軽減するためには,データベースアクセスを厳密に制御する必要がある.
もう1つは生体認証テンプレートの漏洩である.
生体認証はパスワードとは異なり再発行などができないため,漏洩は非常に深刻な問題である.
よって,対策を練ることが大事であるが,生体認証ではユーザー内変動が存在するため,パスワードにおけるセキュリティシステムを用いることはできない.
% 保護スキームはとばす
そこで,生体認証テンプレートを保護する最も簡単な方法として,RSAやAESなどの標準的な暗号化技術を用いてテンプレートを暗号化する手法がある.
しかし生体認証においては,暗号化されたドメインで直接マッチングを行うことができないため,認証試行中には復号化する必要があり,テンプレートを保護する手法としては不十分である.
その問題を克服するための手法として,特徴変換アプローチと生体認証暗号システムが提案されている.

特徴変換アプローチでは,生体認証テンプレートに対して変換関数を適用し,それによって変換されたテンプレートのみがデータベースに保存される.
クエリ特徴にも同じ変換関数が適用され,変換されたテンプレートと直接照合が行われる.
変換関数が可逆的である場合は,攻撃者によって復元される恐れがあるが,ユーザー固有のキーを用いることから,特徴空間でのユーザー間の分離性が向上し,誤一致率が低下するというメリットがある,
% また,変換されたテンプレートの取り消しが容易であることもこの手法が優れている部分である.
変換関数が非可逆的である場合は,攻撃者がテンプレートを復元することが困難であるという明確なメリットがある.
変換のパラメーターが危険にさらされた場合でも,元の生体認証テンプレートを回復するのは難しいため,可逆変換アプローチよりも優れたセキュリティを提供する.
しかし,高い認識パフォーマンスを保持しながら,非可逆的である変換関数を設計することが困難であるというデメリットがある.

生体認証暗号システムは,生体特徴から暗号キーを直接生成する手法であり,例としてキーバインドシステムが挙げられる.
これは,キーと生体認証テンプレートをバインドして,両方の情報を持つ1つの存在をデータベースに保存するシステムである.
ユーザーの生体認証データがなければ,キーまたはテンプレートをデコードすることは困難であり,秘匿性に優れている.

このように生体認証は多くのセキュリティ脅威に対して脆弱であり,その対策が非常に重要となっている.
しかし,すべての要件に対しての完璧な手法はなく,それぞれの脅威に対して適切な手法を用いることが大切であると言える.