\section{最新のバイオメトリクス認証}
一言でバイオメトリクス認証と言っても,様々な種類がある.それらは大きく分けて身体的生体認証と行動的生体認証に分けられる.

まず,最新の身体的生体認証に注目する.
これは個人の持つ身体的特徴を利用して認証するものである.
最近では,スマートフォンの認証に指紋や顔認証が使われるなど,非常に一般的な認証方法になっている.
日常でよく目にするのは指紋,顔による認証であるが,最新の研究では虹彩を利用した技術も開発されている.
これは歩きながらでも,虹彩を高精度かつ迅速に認証するというものである.
これを実現する技術は「歩行者の虹彩を鮮明かつ高解像度に撮影する技術」と「撮影後の画像処理を高速化する画像解析技術」である.
まず,歩行者の目の周辺領域の位置を正確に推定する技術を開発することで,画像データ量を削減し,高解像度に撮影することを実現している.
これにより,利用者がカメラの正面に静止したり,目の位置をカメラに合わせたりするなどの手間を省くことが可能となっている.
また,撮影画像から虹彩認証に適した画質の画像のみを高速に抽出する技術の開発により,瞬時に虹彩認証を行うことを可能としている.
認証の手間を大幅に削減できることから,空港や改札などでの活用が見込まれている.

しかし,こうした身体的生体認証の問題点として,「突破される可能性が比較的高い」ということが挙げられる.
例えば指紋認証においては,画像解析などで指紋を取られる恐れがあり,取られた場合は他人による認証の突破が容易となってしまう.
それだけでなく,指紋や虹彩ではパターンの偽造による攻撃手法も考案されており,安全性の低さが問題視されている側面もある.
さらに身体的生体認証にはプライバシーの問題もある.
身体的生体認証では生体情報を登録する必要があるため,これらの情報が盗まれた場合に,プライバシーが侵害される恐れがある.
このように日常的に使われている身体的生体認証であるが,問題が幾つか存在する.

次に,最新の行動的生体認証に注目する.
行動的生体認証は,デバイス側がユーザーの行動を直接集め,その癖やパターンから認証を行う方法であり,身体的生体認証に比べて「突破されにくく,プライバシー性の高い認証」とされている.
先ほど述べた通り,身体的生体認証では生体情報が盗まれることにより突破される危険がある.
しかし,行動的生体認証における「行動の癖」は非常に複雑なデータから成るため,それらが盗まれたとしても,他人による突破は難しいとされている.
また,行動の癖のデータは盗まれたとしても,個人を特定することは非常に難しく,プライバシーの問題も解決できる.
実際に,歩圧と歩き方のモデルから行動生体認証を可能にする研究などが進められており,非常に高精度であるという結果も得られている.
しかし,行動的生体認証にも「環境や外的要因に大きく作用される」という問題点がある.
つまりその人の置かれている状況(酩酊状態や急いでいる時など)によって行動が変化してしまい,現状の技術ではそのような変化に対応することが難しいということである.
この問題が致命的なこともあり,行動的生体認証は身体的生体認証に比べて普及が進んでいない.

このように,生体認証は日々研究されており,問題点が指摘されつつも,現状のセキュリティシステムとして幅広く使用されている.
2020年の東京オリンピック・パラリンピックではNECの顔認証システムが採用されることが決定されており,生体情報による認証は今後も使用されていくことが予想される.