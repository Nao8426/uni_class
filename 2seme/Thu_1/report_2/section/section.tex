\section*{2.1}
コアとクラッドの屈性率比が1\%であるガラス平面導波路を考える.
波長1.5${\rm \mu m}$の光に対してシングルモード導波路となるコア厚の条件を求めよ.
ガラスの屈折率は1.5とする.
天下りの知識(e.g.,規格化周波数)に依る解答は不可.
あるいは,本章に書かれていない条件式を使う場合は,その条件式の根拠を述べること.

\subsection*{解答}
\noindent
コアの$x$方向への空間的振動の伝搬定数$k_x$とすると
\begin{eqnarray*}
    \sqrt{\frac{(n_1^2-n_2^2)k_0^2}{k_x^2}-1}=\tan\left(k_x\left(\frac{d}{2}\right)+\frac{\pi}{2}\right)
\end{eqnarray*}
$k_x\left(\frac{d}{2}\right)+\frac{\pi}{2}=\pi$のとき,$\sqrt{\frac{(n_1^2-n_2^2)k_0^2}{k_x^2}}=0$なので
\begin{eqnarray*}
    k_x\left(\frac{d}{2}\right)=\frac{\pi}{2}\\
    k_x=\frac{\pi}{d}
\end{eqnarray*}
これを代入して
\begin{eqnarray*}
    \sqrt{\frac{(n_1^2-n_2^2)k_0^2}{\left(\frac{\pi}{d}\right)^2}-1}=0\\
    (n_1^2-n_2^2)k_0^2=\left(\frac{\pi}{d}\right)^2\\
    d^2=\frac{\pi^2}{(n_1^2-n_2^2)k_0^2}
\end{eqnarray*}
$d>0$なので
\begin{eqnarray*}
    d=\sqrt{\frac{\pi^2}{(n_1^2-n_2^2)k_0^2}}
\end{eqnarray*}
$k_0=\frac{2\pi}{\lambda}$より,上式に代入して
\begin{eqnarray*}
    d<\frac{\lambda}{2}\sqrt{\frac{1}{n_1^2-n_2^2}}
\end{eqnarray*}
となる\\
これに$n_1=1.5$,$n_2=\frac{1.5}{1.01}$,$\lambda=1.5$を代入して
\begin{eqnarray*}
    d<3.56[{\rm \mu m}]
\end{eqnarray*}

\section*{2.2}
光ファイバなどの光導波路では,わずかながら光がクラッド領域に漏れ出している.
前問のシングルモード平面導波路において,漏れ出し光の強度が,コアとクラッドの境界面での強度の1/eとなる境界面から距離を求めよ.
ただし簡単のため,$k_z \fallingdotseq n_1k_0$とする($n_1$:コア屈折率,$k_0$:真空中伝搬定数,$k_z$ :伝搬方向の伝搬定数).

\subsection*{解答}
\noindent
境界面が$x=\frac{d}{2}$となるように$x$座標を設定する\\
境界面における光電場$E(x)$は
\begin{eqnarray*}
    E(\frac{d}{2})=A\exp(-\frac{d}{2}\sqrt{k_z^2-n_2^2k_0^2})
\end{eqnarray*}
このときの光強度は
\begin{eqnarray*}
    E^2(\frac{d}{2})=A^2\exp(-d\sqrt{k_z^2-n_2^2k_0^2})
\end{eqnarray*}
境界面での強度の1/eとなる距離を$l$とおくと
\begin{eqnarray*}
    E^2(\frac{d}{2}+l)&=&\frac{1}{e}E^2(\frac{d}{2})\\
    eE^2(\frac{d}{2}+l)&=&E^2(\frac{d}{2})\\
    \exp(1-(d+2l)\sqrt{k_z^2-n_2^2k_0^2})&=&\exp(-d\sqrt{k_z^2-n_2^2k_0^2}\\
    1-(d+2l)\sqrt{k_z^2-n_2^2k_0^2}&=&-d\sqrt{k_z^2-n_2^2k_0^2}\\
\end{eqnarray*}
$k_z^2-n_2^2k_0^2 \neq 0$より
\begin{eqnarray*}
    l&=&\frac{1}{2\sqrt{k_z^2-n_2^2k_0^2}}\\
    &\fallingdotseq&\frac{1}{2\sqrt{n_1^2k_0^2-n_2^2k_0^2}}\\
    &=&\frac{1}{2n_1k_0\sqrt{1-\frac{1}{1.01}}}\\
    &=&\frac{\sqrt{101}}{20n_1k_0}
\end{eqnarray*}

\section*{2.3}
光ファイバの伝播損失は,波長1.55${\rm \mu m}$において,約0.2dB/kmである.
伝播光パワーがファイバ入力時の2\%となる距離はいくらか.
導出過程も記すこと.

\subsection*{解答}
\noindent
求める距離を$l$とおくと,損失は0.2$l$より
\begin{eqnarray*}
    -0.2l&=&10\log{10}(\frac{2}{100})\\
    &=&10(\log{10}2-2)\\
    0.2l&=&20-10\log{10}2\\
    l&=&100-50\log{10}2\\
    &=&85[{\rm km}]
\end{eqnarray*}

\section*{2.4}
本章で述べた現象により,夕焼けは赤く見える.
このことについて説明せよ.

\subsection*{解答}
日中では,波長の短い青い光は細かいチリにぶつかりやすいため,光が散乱する.
このように,青い光が上空で散乱するため青く見える.
一方,太陽が西に沈んでいくと,太陽光はより長い距離の空気層を通過することとなる.
そのため,波長が長い赤い光も細かいチリにぶつかり散乱し始める.
しかし,青い光は波長が短いため届かなくなり,結果的に夕焼けは赤く見える.

\section*{2.5}
本章で述べた現象により,雨上がりの空に虹が見える.
このことについて説明せよ.

\subsection*{解答}
雨上がりでは,大気中に水滴が多く存在する.
その水滴がプリズムの役割をし,太陽光がその水滴の中を通過する際に屈折することで,太陽光が分解されて複数色の帯に見える.
これが虹が見える仕組みである.